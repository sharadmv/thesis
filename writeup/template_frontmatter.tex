%
%
% UCSD Doctoral Dissertation Template
% -----------------------------------
% http://ucsd-thesis.googlecode.com
%
%


%% REQUIRED FIELDS -- Replace with the values appropriate to you

% No symbols, formulas, superscripts, or Greek letters are allowed
% in your title.
\title{Bayesian Structured Representation Learning}

\author{Sharad Vikram}
\degreeyear{\the\year}

% Master's Degree theses will NOT be formatted properly with this file.
\degreetitle{Doctor of Philosophy}

\field{Computer Science}
%\specialization{Machine Learning}  % If you have a specialization, add it here

\chair{Professor Sanjoy Dasgupta}
% Uncomment the next line iff you have a Co-Chair
% \cochair{Professor Cochair Semimaster}
%
% Or, uncomment the next line iff you have two equal Co-Chairs.
%\cochairs{Professor Chair Masterish}{Professor Chair Masterish}

%  The rest of the committee members  must be alphabetized by last name.
\othermembers{
Professor Gary Cottrell\\
Professor Julian McAuley\\
Professor Lawrence Saul\\
Professor Zhuowen Tu\\
}
\numberofmembers{5} % |chair| + |cochair| + |othermembers|


%% START THE FRONTMATTER
%
\begin{frontmatter}

%% TITLE PAGES
%
%  This command generates the title, copyright, and signature pages.
%
\makefrontmatter

%% DEDICATION
%
%  You have three choices here:
%    1. Use the ``dedication'' environment.
%       Put in the text you want, and everything will be formated for
%       you. You'll get a perfectly respectable dedication page.
%
%
%    2. Use the ``mydedication'' environment.  If you don't like the
%       formatting of option 1, use this environment and format things
%       however you wish.
%
%    3. If you don't want a dedication, it's not required.
%
%
%\begin{dedication}
%  To two, the loneliest number since the number one.
%\end{dedication}


% \begin{mydedication} % You are responsible for formatting here.
%   \vspace{1in}
%   \begin{flushleft}
% 	To me.
%   \end{flushleft}
%
%   \vspace{2in}
%   \begin{center}
% 	And you.
%   \end{center}
%
%   \vspace{2in}
%   \begin{flushright}
% 	Which equals us.
%   \end{flushright}
% \end{mydedication}



%% EPIGRAPH
%
%  The same choices that applied to the dedication apply here.
%
%\begin{epigraph} % The style file will position the text for you.
%  \emph{A careful quotation\\
%  conveys brilliance.}\\
%  ---Smarty Pants
%\end{epigraph}

% \begin{myepigraph} % You position the text yourself.
%   \vfil
%   \begin{center}
%     {\bf Think! It ain't illegal yet.}
%
% 	\emph{---George Clinton}
%   \end{center}
% \end{myepigraph}


%% SETUP THE TABLE OF CONTENTS
%
\tableofcontents
\listoffigures  % Comment if you don't have any figures
\listoftables   % Comment if you don't have any tables



%% ACKNOWLEDGEMENTS
%
%  While technically optional, you probably have someone to thank.
%  Also, a paragraph acknowledging all coauthors and publishers (if
%  you have any) is required in the acknowledgements page and as the
%  last paragraph of text at the end of each respective chapter. See
%  the OGS Formatting Manual for more information.
%
\begin{acknowledgements}

\ibhcack

\nvmpack

\loracsack

\solarack
\end{acknowledgements}


%% VITA
%
%  A brief vita is required in a doctoral thesis. See the OGS
%  Formatting Manual for more information.
%
\begin{vitapage}
\begin{vita}
  \item[2014] B.~S. in Electrical Engineering and Computer Science, University of California, Berkeley
  \item[2019] Ph.D. in Computer Science and Engineering, University of California, San Diego
\end{vita}
%\begin{publications}
%  \item Your Name, ``A Simple Proof Of The Riemann Hypothesis'', \emph{Annals of Math}, 314, 2007.
%  \item Your Name, Euclid, ``There Are Lots Of Prime Numbers'', \emph{Journal of Primes}, 1, 300 B.C.
%\end{publications}
\end{vitapage}


%% ABSTRACT
%
%  Doctoral dissertation abstracts should not exceed 350 words.
%   The abstract may continue to a second page if necessary.
%
\begin{abstract}
Bayesian methods offer the flexibility to both model uncertainty and incorporate domain knowledge
into the modeling process.
Deep generative modeling and Bayesian deep learning methods, such as the variational autoencoder (VAE), 
have expanded the scope of Bayesian methods, enabling them to scale to large, high-dimensional datasets.
Incorporating prior knowledge or domain expertise into deep generative modeling is still
a challenge, often resulting in models where Bayesian inference is prohitively slow or even intractable.
In this thesis, I first motivate
using structured priors, presenting a contribution in the space of interactive structure learning.
I then define Bayesian structured representation learning (BSRL) models, which combine
structured priors with the VAE, and present
foundational work, along with applications of BSRL models.
\end{abstract}


\end{frontmatter}

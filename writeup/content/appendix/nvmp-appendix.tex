\section{Proof of ELBO lower bound}
\label{sec:nvmp-proof}

\begin{proof}
(This proof is very similar to that in \citet{Johnson2016}.)
In each step of mean-field variational inference,
we maximize the ELBO with respect to a single variational factor.
Consider a latent variable $\z_\n$.
Choosing a form for $\q(\z_\n)$
that does not capture the optimal $\q^*(\z_\n)$
will result
in a lower maximum than the maximizing
over all possible $\q(\z_\n)$ distributions.
We express this as
\begin{equation}
    \max_{\q(\z_\n)}\L[\q] \ge\q  \L(\eta_{\z_\n}^*)
\end{equation}
where 
\[
\eta_{\z_\n}^* \triangleq \argmax_{\eta_{\z_\n}} \L[\q]
\]
is the optimal parameter of some chosen form of $\q(\z_\n)$.

In NVMP, we further modify this local optimization
by modifying the ELBO to include recognition networks
using the trick from \cite{Johnson2016}.
This results in a surrogate objective $\hat{\L}_\recparam[\q]$,
and maximizing this objective can do no better than maximizing the
parameterized objective $\L[\q]$.
\begin{equation}
    \max_{\q(\z_\n)}\L[\q] \ge\q  \L(\eta_{\z_\n}^*) \ge\q
    \L(\hat{\eta}_{\z_\n}^*)
\end{equation}
where 
\[
\hat{\eta}_{\z_\n}^* \triangleq \argmax_{\eta_{\z_\n}} \hat{\L}_\recparam[\q]
\]

As each local optimization lower-bounds the ELBO,
the NVMP optimization holistically 
lower-bounds the ELBO.

This bound is tight in that
if there exists a set of recognition
network weights $\recparam$ such that
\begin{equation}
\log \p(c \given \z_\n, \parents_c \not{\z_\n}) = \psi(c, \z_\n, \parents_{c} \not{\z_\n}; \recparam) \\
\end{equation}
we have
\begin{equation}
    \max_{\q(\z_\n)}\L[\q] \geq  \L(\eta_{\z_\n}^*) = 
    \L(\hat{\eta}_{\z_\n}^*)
\end{equation}
\end{proof}

\section{Extension: stochastic variational inference}
\label{sec:nvmpsvi}
SVI \citep{Hoffman2013} treats local and global variational factors differently,
first performing VMP to optimize local
factors to completion. Note that these factors are only
optimized over mini-batches of data.
Global variational factors are then updated
with natural gradients $\tilde{\nabla}_{\globals}\L$ of the ELBO, 
which take a simple form in conjugate exponential models. Specifically,
for the natural parameter for the global variables
$\eta_{\globals}^\q$,
the natural gradient update is
\begin{equation}
    \tilde{\nabla}_{\eta^\q_{\globals}} \L =  f_{\globals}\left(\{\vmpmessage{p}{\globals}\}_{p \in \pi_{\globals}}\right) + B \left(\sum_{c \in \children_{\globals}}\vmpmessage{c}{\globals}\right) - \eta^\q_{\globals}
\end{equation}
where $B$ is the number of minibatches in the dataset.

Importantly,
this natural gradient update natural gradient can be computed from VMP messages.
Thus, integrating SVI into NVMP requires no extra steps, 
as NVMP simply modifies the messages.
NVMP can thus be treated as the inner-loop
in stochastic VMP, where natural gradients are computed
from NVMP messages when necessary. \footnote{This results in natural gradients computed over the NVMP surrogate objective.}

\section{Extension: amortized inference}
\label{sec:amortizeddetail}
NVMP produces approximates previously messages. 
Thus, the role
of the recognition network is to assist in inference,
rather than directly amortizing inference.
This distinction is analogous to the difference
between the VAE and SVAE. In the VAE,
the recognition network directly outputs
the model parameters of the latent variables,
whereas in SVAE, the recognition network
outputs a message. 
This difference plays a part in practice, 
when we may desire a network that directly acts as a variational posterior,
avoiding the use of message passing in answering an inference query.

NVMP can be modified to include
recognition networks that directly output parameters.
Previously, where the optimal natural parameter for a factor $\q(\z_\n)$
was computed as a combination of messages
\begin{equation}
   \eta^\q_{\z_\n} = f_{\z_\n}\left(\{\vmpmessage{p}{\z_\n}\}_{p \in \parents_{\z_\n}}\right) + \sum_{c \in \children_{\z_\n}} \vmpmessage{c}{\z_\n}
\end{equation}
we replace the entire natural parameter with the output of a neural network
that takes children and coparent samples as input
\begin{equation}
   \eta^{\text{amort.}}_{\z_\n} = r^{\z_\n}_\recparam(\left\{c'\sim \q(c)\right\}_{c \in \children_{\z_\n}}, \{v' \sim \q(v)\}_{v \in \parents_c\not{\z_\n}})
\end{equation}
This produces a VAE-style algorithm, where the recognition networks
directly produce parameters and thus have utility outside of message passing.
However, we do lose the coordinate-ascent style of the algorithm, as
$\eta^{\text{amort.}}_{\z_\n}$ is not the optimal parameter
of a surrogate objective, but the natural parameter of a variational
distribution parameterized by $\recparam$.

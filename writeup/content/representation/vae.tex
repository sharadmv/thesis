\chapter{Representation Learning}

Two of the failure modes of structure learning
algorithms are
characteristic
of modern datasets.
The first is that structure learning
algorithms often struggle to 
model high-dimensional data.
Consider applying
fitting a Gaussian mixture
model (GMM) to an image dataset.
The likelihood model in a GMM
assumes data are sampled
from the multivariate Gaussian
distribution, which is
often too simple of a likelihood
to model complex, high-dimensional data like images.
The second problem
is that of computation.
Fitting a Gaussian mixture model
to millions of high-dimensional data
will require inverting
large matrices, which will
easily be the bottleneck in a
learning algorithm.
However, the era of deep learning has
been defined by success in machine learning
tasks on
complex datasets
like ImageNet \citep{Deng2009} and CIFAR \citep{Krizhevsky2009}
through the efficient use of deep neural networks.
Adapting these bigger models
to
unsupervised learning has 
introduced new challenges.

In this chapter, we discuss the problem
of representation learning.
Consider the problems that arise 
with structure learning in high dimensions.
If we were able to instead embed 
high-dimensional data into a lower-dimensional space
that retains most, if not all,
of its semantic information,
why not work with the embeddings instead?
This question encompasses the goal
of representation learning, namely
learning \emph{useful} representations
for downstream tasks.
Such downstream tasks may
be structure learning, classification,
or even reinforcement learning.
In this chapter, we
discuss methods for representation learning
leading up to the variational autoencoder (VAE),
which is the key preliminary
for the remainder of the thesis.

\section{PCA and autoencoding}
Two of the simplest and most intuitive
representation learning algorithms
are principle component analysis (PCA)
and autoencoding.
PCA takes a $\d$-dimensional dataset 
$\dataset \in \R^{\numdata \times \d}$ and
reduces it to a $\l$-dimensional
dataset $\latentdataset \in \R^{\numdata \times \l}$
where usually $\l \ll \d$.
PCA finds a $\l$-dimensional
linear subspace,
in contrast with autoencoders,
which learn a nonlinear subspace.
Autoencoders
minimize reconstruction loss
through an encoder
and decoder deep neural network.
Specifically, we construct
an encoder network $\encoder_\recparam(\x)$
and a decoder network $\decoder_\genparam(\z)$
and minimize reconstruction error
\begin{align*}
\recparam^*, \genparam^* = \argmin_{\genparam, \recparam} \sum _{\n = 1}^\numdata \L(\x_\n, \decoder_\genparam(\encoder_\recparam(\x_\n)))
\end{align*}
We can train autoencoders
with minibatch stochastic gradient descent,
enabling us to 
minimize this loss function
even with large amounts of data.
The learned representations 
$\latentdataset$ are just the dataset $\dataset$
passed through the encoder network.

\section{The variational autoencoder}
The variational autoencoder \citep[VAEs; ][]{Kingma2014, Rezende2014}
formulates autoencoders probabilistically.
Specifically, it
assumes
latent codes $\latentdataset$
are sampled from a
prior distribution $\p(\latentdataset)$.
The latent codes
are then independently passed through a stochastic decoder
$\p(\x_\n \given \z_\n) \triangleq \p(\decoder_\genparam(\z))$,
where $\decoder_\genparam(\z_\n)$ outputs
the parameters of a distribution over $\x_\n$.
This generative model is visualized in
\autoref{fig:graphical-model-rep}.
Representation learning in
this probabilistic formulation amounts to
computing the posterior distribution 
$\p(\latentdataset \given \dataset)$.

The VAE uses
neural network encoders and decoders,
but the same formalization
can be applied to PCA.
Specifically, we
use linear Gaussian decoder,
i.e. $\decoder_\genparam(\z_\n) = (\dynmat \z_\n, \variance I)$
\begin{align*}
    \z_\n &\sim \N(0, I) \\
    \x_\n \given \z_\n &\sim \N(\decoder_\genparam(\z_\n)) \triangleq \N(\dynmat \z_\n, \variance I)
\end{align*}
This model, named probabilistic PCA (PPCA)
generalizes PCA in that the two models
are equivalent as $\variance \rightarrow 0$.
Formulating PCA probabilistically
is helpful because it concretizes
the assumptions that we
put into our representations.
Assuming an isotropic normal
prior encourages $\latentdataset$
to roughly be zero-centered and
have independent dimensions.
Similarly, the linear-Gaussian
likelihood model means
that our observed data
should be Gaussian with
a low-rank covariance matrix structure.
PPCA, however, is often too simple
of a model for complex data like images,
especially if the task is generating
realistic images. Neural networks, on
the other hand, can capture complex
dependencies in high-dimensional
data and are a better fit
for such data. 

The VAE replaces the linear Gaussian
decoder with a neural
network decoder.
The neural network decoder outputs
the parameters of a Gaussian distribution,
which is then sampled to observe $\x_\n$.
\begin{align*}
    \z_\n &\sim \N(0, I) \\
    \x_\n \given \z_\n &\sim \N(\decoder_\genparam(\z_\n)) \triangleq \N(\mu_\gamma(\z_\n), \Sigma_(\z_\n))
\end{align*}

The VAE is a logical extension to PPCA,
replacing simple linear-Gaussian
likelihood model with a high capacity
neural network. A large enough 
neural network could plausibly
generate realistic looking images
 text, or other complex data
from a low dimensional representation $\z_\n$,
given enough training data.
However, the simple intuition that comes
with PPCA and linear models 
is lost and we no longer
have a tractable marginal distribution over data.
Furthermore, the choice of prior
distribution $\p(\latentdataset)$ has critical importance,
due to its large influence on the learned latent space.

The VAE is a logical extension to PPCA, but
introduces a challenge in posterior inference.
A neural network decoder makes the posterior
analytically intractable, but also
introduces computational issues that make traditional
variational inference and MCMC prohibitively slow.
Cleverly, \cite{Kingma2014} use a
neural network to approximate the posterior distribution
$\p(\z_\n \given \x_\n)$. They introduce
a recognition network $\encoder_\recparam$,
which outputs the parameters of a distribution
over $\z_\n$ (typically Gaussian). This
approximate posterior distribution $\q_\recparam(\z_\n \given \x_\n) \triangleq \N(\encoder_\recparam(\x_\n))$ is then optimized to be 
close to the true $\p(\z_\n \given \x_\n)$
using a method called variational inference.

\subsection{Variational inference}
The term \emph{variational inference}
encompasses algorithms
that frame Bayesian inference
as an optimization
problem. In the case
of latent variable models
\footnote{We focus on
single examples $\x$ and $\z$ for notational
simplicity},
we have a posterior distribution $\p(\z \given \x)$ we wish to
infer, which is
usually analytically intractable.
We introduce a variational approximation
$\q_\recparam(\z)$ where $\recparam$
are free parameters to be optimized.
Variational inference focuses on the 
optimization problem 
\begin{align*}
\recparam^* = \argmin_\recparam \KL{\q_\recparam(\z)}{\p(\z \given \x)}
\end{align*}

Note that variational inference optimizes $\KL{q}{p}$
rather than $\KL{p}{q}$, which is a different objective
since KL-divergence is an asymmetric function.
Since $\KL{q}{p}$ has an expectation with respect
to $\q_\recparam(\z)$, KL
is high when $q_\recparam(\z)$ is large and $\p(\z \given \x)$ is
small, and is low when $q_\recparam(\z)$ is small
regardless of $\p(\z \given \x)$. This means that
variational inference encourages $\q(\z)$ to
fit a mode of $\p(\z \given \x)$ rather than trying to 
cover all of its support.

Directly minimizing $\KL{q}{p}$ is typically
also intractable, since it directly 
includes the already intractable posterior
$\p(\z \given x)$. However, we can 
massage it into a form that is
more amenable to optimization.

\begin{align*}
    \KL{\q_\recparam(\z)}{\p(\z \given \x)} &=  
    -\E_\q \left[\log\frac{\p(\z \given \x)}{\q_\recparam(\z)}\right] \\
    &= -\E_\q\left[\log \frac{\p(\x)\p(\z \given \x)}{\p(\x)\q_\recparam(\z)}\right] \\
    &= -\E_\q\left[\log \frac{\p(\x, \z)}{\p(\x)\q_\recparam(\z)}\right] \\
    &= \log \p(\x) -\E_\q\left[\log \frac{\p(\x, \z)}{\q_\recparam(\z)}\right] \\
    &= \log \p(\x) - \L[\q_\recparam]
\end{align*}

The KL breaks down into two important terms.
These first term, $\log \p(\x)$
is the marginal data likelihood,
and is usually intractable to compute directly
since it involves integrating over all possible
values of $\z$.
The term on the right, however,
is tractable to compute since 
it is a expectation of known quantities,
specifically the joint distribution $\p(\x, \z)$
and the variational approximation $\q_\recparam(\z)$. 
The left and right hand side are both lower bounded by zero,
since $\KL{\q_\recparam(\z)}{\p(\z \given \x)} \geq 0$,
and when
it is zero,
\begin{align*}
    \KL{\q_\recparam(\z)}{\p(\z \given \x)} = 0 \iff \log \p(\x) &= \L[\q_\phi]
\end{align*}
$\L[\q_\recparam]$ is called the
\emph{evidence lower bound} or ELBO for short.
It is a lower bound on the marginal data
likelihood $\log \p(\x)$, and is equal 
to it when $\KL{\q_\recparam(\z)}{\p(\z \given \x)} = 0$.
The goal of variational inference is to maximize the ELBO with respect to $\recparam$, or
\begin{align*}
\recparam^* = \argmax_\recparam \L[\q_\recparam] \triangleq \E_\q\left[\log \frac{\p(\x, \z)}{\q_\recparam(\z)}\right] 
\end{align*}
Unlike directly optimizing $\KL{\q_\recparam(\z)}{\p(\z \given \x)}$, optimizing the ELBO is usually tractable, but algorithms
depend on the model and choice of variational posterior.
For a more detailed explanation of variational inference,
please refer to \citet{Blei2017}.

\subsection{Inference in the VAE}
The VAE uses an approximate posterior
parameterized by a neural network.
Specifically, it uses conditional 
posterior distribution where the
parameters are the output of $\x_\n$
passed through a neural network,
or
$\q_\recparam(\z \given \x) \triangleq \N(\encoder_\recparam(\x))$.\footnote{We use the normal distribution for the encoding distribution for the entirety of this thesis.}
The free parameters in the VAE 
variational posterior are the
encoder weights $\recparam$ 
and the likelihood term
also includes decoder weights $\genparam$, resulting
in the following optimization problem:
\begin{align*}
\recparam^*,\genparam^* = \argmax_{\genparam, \recparam} \L[\q_\recparam] &\triangleq \E_\q\left[\log \frac{\p(\z)\p_\genparam(\x \given \z)}{\q_\recparam(\z \given \x)}\right] \\
&= \E_\q \left[ \log \p_\genparam(\x \given \z)\right] - \KL{\q_\recparam(\z \given \x)}{\p(\z)}
\end{align*}
When breaking down the ELBO in this way,
we get an autoencoding-like objective function.
The first is a log-likelihood term for data, measuring how well the observed data is reconstructed.
$\E_\q \left[ \log \p_\genparam(\x \given \z)\right]$ is 
the expected likelihood of a data point w.r.t.
our variational posterior, but it cannot be computed directly
since it involves integrating over
$\z$. It can, however, 
be approximated with Monte Carlo sampling, i.e. $\E_\q \left[ \log \p_\genparam(\x \given \z)\right] \approx \frac{1}{S}\sum_{s=1}^S \log \p_\genparam(\x \given \z^{(s)})$ where $\{\z^{(s)}\}_{s = 1}^S \sim \q(\z \given \x)$.
In practice, we set $S = 1$, since the variance of gradients
of the ELBO can also be lowered by larger minibatches.

The second term of the ELBO is the KL divergence of the posterior distribution to the prior. This intuitively acts as
a regularizer on the latent space. Consider
an isotropic normal prior $\z \sim \N(0, I)$. The KL
term encourages the posterior $\q_\recparam(\z \given \x)$
to thus look like an isotropic normal, 
at the very least preventing it from collapsing it to
a point. 
The VAE, at some level, acts as regularized autoencoder, though understanding theoretical properties of the ELBO
is still an active area of research \citep{Burda2015, Hoffman2016, Alemi2017}.

\begin{figure}[htp!]
    \centering
    \includegraphics{tikz/representation/rep}
    \caption{The graphical model for representation learning}
    \label{fig:graphical-model-rep}
\end{figure}

We can optimize the free parameters of the VAE ELBO
using stochastic gradient descent
using a Monte Carlo approximation of the likelihood term.
\begin{align*}
\L[\q_\recparam] &= \E_\q \left[ \log \p_\genparam(\x \given \z)\right] - \KL{\q_\recparam(\z \given \x)}{\p(\z)} \\
                 &\approx \frac{1}{S}\sum_{s = 1}^S \log \p_\genparam(\x \given \z^{(s)}) - \KL{\q_\recparam(\z \given \x)}{\p(\z)}
\end{align*}
The KL term can usually be computed in closed form (if both $\q(\z \given \x)$ and $\p(\z)$ are Gaussian, for example), or can be approximated
via Monte Carlo. We can compute gradients of the approximate
ELBO w.r.t. $\genparam$ and $\recparam$, and use a
first-order optimization to find local maximum. \citet{Kingma2014} and \citet{Rezende2014}
popularized the \emph{reparametrization trick} for VAEs,
where in samples $\{z^{(s)}\}_{s = 1}^S \sim \q(\z \given \x)$ are generated
via a deterministic transformation of independent noise.
For example, if $\q(\z \given \x)$ is a Gaussian distribution (parameterized
by a neural network), samples can be generated
by first sampling independent isotropic Gaussian noise $\epsilon^{(s)} \sim \N(0, I)$, then transforming the noise according to the parameters
of $\q(\z \given \x) = \N(\mu, \Sigma)$
\begin{align*}
    \z^{(s)} = \mu + L\epsilon^{(s)}
\end{align*}
where $L = \chol{\Sigma}$. The reparameterization trick
has empirically resulted in lower variance gradients
when compared to alternatives
like BBVI/REINFORCE/score function estimators \citep{Ranganath2014, Williams1992}.

\subsection{Extensions and applications of the VAE}
The simplicity and intuitiveness of the
VAE has led to follow up research
in several directions, of which we
will discuss a couple.

%\subsubsection{Autoregressive and recurrent decoders}
One line of work has explored
decoders that result have powerful
likelihoods over images.
In the vanilla VAE, an observed image
is a sample 
from a Gaussian 
or Bernoulli distribution
parameterized by a neural network.
Pixels are sampled independently,
though the parameters are coupled
through the neural network.
Recurrent models like 
DRAW \citep{Gregor2015}
sample the output image
through a recurrent
neural network that outputs
the image conditionally.
Other autoregressive decoders
like PixelVAE \citep{Gulrajani2016}
generate an image with conditional
convolution
operations.
These decoders, although
computationally more
expensive than an independent likelihood,
have demonstrated great improvement in
generated image quality and 
held-out log-likelihood scores.

Another line of work
has explored variations
on the training objective
for the VAE that result in
either better representations
or a better lower bound on the
ELBO. The importance weighted autoencoder \citep{Burda2015}
explores a tighter lower bound by means
of importance sampling and also
explores the problem of variational
over-pruning, wherein
dimensions of the latent space
are set to match the prior distribution
in order to minimize the KL term of the ELBO.
\citet{Hoffman2017} mitigates
this problem by training the generator
with an MCMC objective, rather than the direct
ELBO. Other works explore the trade-off
between the terms in the ELBO. The $\beta$-VAE \citep{Higgins2017}
introduces a scalar weight on the KL term of the ELBO,
and \citet{Alemi2017} applies
an information theoretic perspective to the tradeoff
between the reconstruction KL term.
